Eventos e, mais geralmente, variáveis aleatórias crescentes do modelo de percolação
têm a propriedade de serem positivamente correlacionadas.
 
\vfill\eject

\bteo[Desigualdade de FKG]
\label{teo:fkg}
Sejam $Z$ e $Y$ duas variáveis a\-le\-a\-tó\-rias crescentes e limitadas em $\Om$.
Então
\beq
\ep(ZY)\geq\ep(Z)\ep(Y).
\eeq
\eteo

\vs

\noindent{\bf Prova do Teorema \ref{teo:fkg}}

Vamos supor inicialmente que $Z$ e $Y$ sejam cilíndricas, isto é, dependam apenas 
de um conjunto finito de
elos $\{e_1,e_2,\ldots,e_n\}$. Provaremos o teorema neste caso por indução em
$n$.

Para $n=1$, $Z=f(X_{e_1})$ e $Y=g(X_{e_1})$, onde $f$ e $g$ são crescentes.
Seja $Y'$ uma cópia independente de $X_{e_1}$ (isto é, $Y'$ e $X_{e_1}$ são
i.i.d.). Então
$$[f(X_{e_1})-f(Y')][g(X_{e_1})-g(Y')]\geq0,$$
pelo fato de $f$ e $g$ serem crescentes. Portanto
$$\ep\left\{[f(X_{e_1})-f(Y')][g(X_{e_1})-g(Y')]\right\}\geq0.$$
Expandindo o termo à esquerda, temos
\beqn\nonumber&&\ep[f(X_{e_1})g(X_{e_1})]+\ep[f(Y')g(Y')]\\ \label{eq:f1}\ge
\ep[f(X_{e_1})g(Y')]+\ep[f(Y')g(X_{e_1})].\eeqn Pela independência entre 
$X_{e_1}$ e $Y'$, a expressão à direita fica
$$\ep[f(X_{e_1})]\ep[g(Y')]+\ep[f(Y')]\ep[g(X_{e_1})].$$
Como $X_{e_1}$ e $Y'$ têm a mesma distribuição, a desigualdade (\ref{eq:f1})
fica
\beqnn&&\ep[f(X_{e_1})g(X_{e_1})]+\ep[f(X_{e_1})g(X_{e_1})]\\ \ge
\ep[f(X_{e_1})]\ep[g(X_{e_1})]+\ep[f(X_{e_1})]\ep[g(X_{e_1})],\eeqnn
isto é
$$2\ep[f(X_{e_1})g(X_{e_1})] \geq
2\ep[f(X_{e_1})]\ep[g(X_{e_1})]$$
e o resultado está provado para $n=1$. 

Supondo-o válido para $n=k$, seja $n=k+1$. Então 
$$Z=f(X_{e_1},\ldots,X_{e_k},X_{e_{k+1}})\quad\mbox{e}\quad
Y=g(X_{e_1},\ldots,X_{e_k},X_{e_{k+1}}),$$ com $f$ e $g$ crescentes.

Agora
\beqnn
\ep(ZY)\=\ep\left[f(X_{e_1},\ldots,X_{e_k},X_{e_{k+1}})
g(X_{e_1},\ldots,X_{e_k},X_{e_{k+1}})\right]\\
\=\ep\left\{\ep\left[f(X_{e_1},\ldots,X_{e_k},X_{e_{k+1}})
g(X_{e_1},\ldots,X_{e_k},X_{e_{k+1}})|X_{e_{k+1}}\right]\right\}.
\eeqnn
Na esperança condicional acima, $X_{e_{k+1}}$ está fixo e portanto
$f$ e $g$ podem ser vistas como funções de $X_{e_1},\ldots,X_{e_k}$
e a hipótese de indução pode ser aplicada para dar que a última 
expressão acima é maior ou igual a 
$$\ep\left\{\ep\left[f(X_{e_1},\ldots,X_{e_k},X_{e_{k+1}})|X_{e_{k+1}}\right]
\ep\left[g(X_{e_1},\ldots,X_{e_k},X_{e_{k+1}})|X_{e_{k+1}}\right]\right\}.$$
Agora é claro que as esperanças condicionais acima são funções crescentes
de $X_{e_{k+1}}$. Novo uso da hipótese de indução produz o resultado para
$n=k+1$, completando o passo de indução.

Para completar a demonstração, consideremos $Z$ e $Y$ não necessariamente
cilíndricas. Seja $e_1,e_2,\ldots$ uma enumeração de $\ed$. Pelo Teorema da
Convergência de Martingais (veja \cite{kn:B}),
$$Z=\lim_{n\to\infty}\ep\left[Z|X_{e_1},\ldots,X_{e_n}\right]$$
e de maneira semelhante para $Y$.
Pelo passo anterior, a desigualdade de FKG vale quando $Z$ e $Y$ são substituidos
por $$\ep\left[Z|X_{e_1},\ldots,X_{e_n}\right]\quad\mbox{e}\quad 
\ep\left[Y|X_{e_1},\ldots,X_{e_n}\right].$$ Uma passagem ao limite em $n$ e o Teorema
da Convergência Dominada nos dão o resultado geral. $\bo$

\vs

\bcor
\label{cor:fkg}
Se $A$ e $B$ forem eventos crescentes, então
\beq
\label{eq:fkg}
\p(A\cap B)\geq\p(A)\p(B).
\eeq
\ecor

\vs

\noindent{\bf Prova}

Basta aplicar o Teorema \ref{teo:fkg} com $Z=\ind_A$ e $Y=\ind_B$ $\bo$

\vs

Observamos que a desigualdade em (\ref{eq:fkg}) é equivalente a
$$\p(A|B)\geq\p(A),$$ 
o que nos diz que a ocorrência de um evento crescente aumenta a probabilidade
de ocorrência de um outro evento crescente.

\vs

O Teorema~\ref{teo:fkg} vale também para duas v.a.'s 
decrescentes (na ordem parcial), pois a desigualdade não muda se substituirmos $Z$ e $Y$ respectivamente
por $-Z$ e $-Y$, que são crescentes. Segue que o Corolário~\ref{cor:fkg} vale
também para dois eventos decrescentes.

\vs

As desigualdades acima foram primeiro provadas por Harris \cite{kn:H} e
posteriormente generalizadas para outros modelos por Fortuin, Kasteleyn e
Ginibre \cite{kn:FKG}, cujas iniciais batizaram-nas.