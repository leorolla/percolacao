Este capítulo será dedicado à apresentação de alguns resultados auxiliares 
a serem utilizados nos capítulos subseqüentes.
Todos dizem respeito a funções e eventos {\em crescentes}, que passamos a definir.

Para isto, introduzimos a ordem parcial em $\Om$
$$\om\leq\om'\Leftrightarrow\ \om_e\leq\om'_e\,\,\mbox{para todo $e\in\ed$}.$$ 

Uma variável aleatória $X$
é dita crescente se for crescente na ordem parcial acima, isto é,
$$X(\om)\leq X(\om')\,\,\mbox{sempre que}\,\,\om\leq\om'.$$ 
Um evento $A\in{\cal E}$ é dito crescente se $\ind_A$, a função indicadora
de $A$, for crescente.

Em palavras, um evento $A$ é crescente sempre que para cada configuração de
elos abertos em que $A$ ocorre, ao abrirmos mais elos nesta configuração,
$A$ continua ocorrendo. Exemplos comuns são os eventos $\{x\lr y\}$ em 
que dois sítios da rede estão conectados por um caminho de elos abertos
e $\{\cl=\infty\}$ em que o aglomerado da origem é infinito. 