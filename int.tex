Percolação é o fenômeno de transporte de um fluido através de um
meio poroso. Por exemplo, óleo ou gás através da rocha ou água
através de pó de café. O meio é constituido de poros e canais
microscópicos por onde passaria o fluido. Numa situação simples, 
cada canal pode estar aberto ou fechado à passagem do fluido, dependendo 
de diversas características que poderiam ser resumidas num parâmetro.
A distribuição de canais abertos e fechados poderia ser descrita
probabilisticamente. No caso mais simples, cada canal, independentemente
dos demais, está aberto com probabilidade $p$, o parâmetro do modelo, e fechado
com a probabilidade complementar.
Vamos modelar o meio microscopicamente pelo reticulado
hipercúbico $d$-dimensional,
os sítios do reticulado representando os poros e os elos representando os canais.
Este, o que chamaremos de {\em modelo de percolação de elos independentes (em $\zd$)},
será o objeto do nosso estudo.
A questão básica é a ocorrência ou não de {\em percolação}, isto é,
a existência de um caminho infinito de elos abertos atravessando o meio.
A seguir, introduziremos o modelo em detalhe (na próxima seção) e mostraremos
(na seção seguinte)
seu primeiro resultado não-trivial, aquele que estabelece a transição de fase em
duas ou mais dimensões, isto é, a existência de um valor crítico não-trivial
para o parâmetro $p$, abaixo do qual o modelo não exibe percolação e acima do
qual esta passa a ocorrer.
