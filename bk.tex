A próxima desigualdade que discutiremos vai no sentido contrário da
desigualdade de
FKG e envolve uma intersecção {\em restrita} de eventos crescentes.

Dados dois eventos crescentes $A$ e $B$ de ${\cal E}$, dizemos que $A$ e $B$ ocorrem 
{\em disjuntamente} (para um dado $\om$) se existirem dois caminhos abertos (de elos)
disjuntos (em $\om$) tais que o primeiro garante a ocorrência de $A$ e o segundo 
garante a ocorrência de $B$.
Denotamos por $A\circ B$ a ocorrência disjunta de $A$ e $B$.

Por exemplo, no evento $\{x\lr y\}\circ\{u\lr v\}$ há dois caminhos abertos disjuntos,
um ligando os sítios $x$ e $y$ e outro ligando os sítios $u$ e $v$.

\vs

\bteo[Desigualdade de BK]

Sejam $A$ e $B$ dois eventos crescentes de ${\cal E}$
dependendo apenas de um conjunto finito de elos. Então
$$
\p(A\circ B)\leq\p(A)\p(B).
$$
\eteo

\vs

O nome da desigualdade é referência a seus descobridores van den Berg e 
Kesten \cite{kn:vBK}.
A restrição a eventos que dependem apenas de um conjunto finito de elos
deve-se a
razões técnicas, a extensão pode ser feita para outros casos de interesse.
Para uma discussão mais completa com a demonstração do resultado, veja
\cite{kn:G} página 29. Discutiremos abaixo a idéia da prova, usando os eventos
do exemplo acima, mas restritos a uma sub-rede finita de $\zd$ (por exemplo, o quadrado 
$Q_M$ do capítulo anterior, para algum $M$), do contrário eles dependerão de um 
conjunto infinito de elos.

Notemos para começar que dada a ocorrência de $\{u\lr v\}$, temos informação
sobre elos abertos, que {\em não podem} ser usados na ocorrência disjunta de
$\{x\lr y\}$. Isto torna plausível a desigualdade
$$\p(\{x\lr y\}\circ\{u\lr v\}|u\lr v)\leq\p(x\lr y).$$

A idéia da prova é a seguinte. Seja ${\cal G}$ uma sub-rede finita de $\zd$ e $e$ um
elo de ${\cal G}$. Substituamos $e$ por dois elos paralelos 
$e'$ e $e''$ abertos com probabilidade $p$ e fechados com probabilidade $1-p$
independentemente um do outro. Considera-se a ocorrência disjunta de 
$\{x\lr y\}$ e $\{u\lr v\}$ nesta nova rede, mas com o primeiro evento evitando
$e''$ e o segundo evento evitando $e'$. Observa-se que esta operação não pode
diminuir a probabilidade original. Continua-se
indutivamente substitindo-se os elos $f$ de ${\cal G}$ por elos paralelos 
e independentes
$f'$ e $f''$ e considerando-se a ocorrência disjunta de $\{x\lr y\}$ e $\{u\lr v\}$
na nova rede, a ocorrência do primeiro sem usar elos $''$ e a do segundo
sem usar elos $'$. A operação não diminui a probabilidade do passo anterior. 
Ao término, esgotados todos os elos de ${\cal G}$, temos
duas cópias independentes desta rede, uma na qual perguntamos pela ocorrência de
$\{x\lr y\}$, na outra perguntamos pela ocorrência de
$\{u\lr v\}$, eventos portanto independentes. A probabilidade final é então o produto
das probabilidades dos eventos e a cadeia de desigualdades levando à probabilidade
de ocorrência disjunta dos eventos em ${\cal G}$ nos dá o resultado.

\begin{defin}
\label{def:con}
À probabilidade de que dois sítios $x$ e $y$ estejam conectados por um caminho
aberto, $$\p(x\lr y),$$
chamamos {\em função de conectividade} (entre $x$ e $y$), com a notação
$\tp(x,y).$
\end{defin}