Dos modelos da física estatística na rede a exibir transição de fase,
o modelo de Percolação é possivelmente o mais simples e um dos que
mais bem
exemplificam a rica e frutífera interrelação que há na área entre
métodos da física matemática, probabilidade e combinatória.

Formulado em fins da década de 50 por Broadbent e Hammersley \cite{kn:BH} como
um modelo de transporte de fluido em meio poroso, ele teve seus 
primeiros resultados não-triviais (sobre a existência de transição 
de fase) provados por estes autores. Harris \cite{kn:H} obteve resultados parciais
sobre o ponto crítico em duas dimensões no início dos anos 60. Mais 
tarde, já em fins dos anos 70 e início dos 80, Kesten \cite{kn:K1} 
estabeleceu seu valor exato. 
Diversos outros resultados importantes foram obtidos neste último 
período, como os argumentos independentes de Menshikov \cite{kn:M} 
e Aizenman e Barsky \cite{kn:AB} para estabelecer a unicidade do ponto crítico
e o resultado de Aizenman, Kesten e Newman sobre a unicidade do aglomerado infinito \cite{kn:AKN}.

Os fins dos anos 80 e início dos 90 marcam o ataque a um dos problemas mais 
elusivos do modelo, a continuidade da densidade do aglomerado infinito no 
ponto crítico em 
mais do que duas dimensões. Idéias de renormalização de Barsky, Grimmett e
Newman \cite{kn:BGN} produziram os resultados mais importantes a respeito, ainda que
incompletos (o problema original permanece em aberto!). 

\vspace{1cm}

Estas notas representam tópicos apresentados pelo autor em cursos
sobre Percolação na USP de São Carlos e São Paulo, na UFMG e no 
IMPA entre
janeiro de 1993 e fevereiro de 1994. Os pontos abordados são basicamente os
delineados acima.
As fontes são o livro já bastante 
aclamado {\em Percolation} 
de G.R. Grimmett \cite{kn:G}, que serve de referência para tudo aqui e muito mais,
e também notas de aulas tomadas de C.M. Newman na NYU em 1990. 
Supõe-se um conhecimento de teoria da probabilidade a nível de graduação. 
Alguns resultados mais avançados
(mas clássicos) desta teoria são citados, para os quais indicamos, por exemplo,
Breiman \cite{kn:B} como referência.

Agradeço o coleguismo e amizade dos mentores dos cursos que mencionei, 
Cláudio Paiva, Gastão Braga e Maria Eulália Vares.
Agradecimentos especiais a esta última pela iniciativa de sugerir 
e organizar a edição destas notas junto ao IMPA/CNPq.

\begin{flushright}
julho de 1996
\end{flushright}


