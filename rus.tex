O próximo resultado é uma fórmula para a derivada em $p$ da probabilidade de
um evento crescente. Para obtê-la, usaremos a construção acoplada do modelo
de percolação
usando a família de variáveis uniformes ${\cal Z}$ vista no capítulo anterior
(o modelo padrão).

Consideramos então, para uma dada configuração das variáveis em ${\cal Z}$, a
configuração de elos $p$-abertos $\op$, isto é, $(\op(e))_{e\in\ed}$ tal que 
$$\op(e)=\cases{1,& se $Z_e<p$\cr
                0,& caso contrário}$$
para todo $e\in\ed$.

Seja $A$ um evento crescente que depende de um conjunto finito de elos $\g$ de
$\ed$ e considere
\beq
\label{eq:r1}
\ppd(A)-\p(A)=\pp(\op\notin A, \opd\in A).
\eeq

Se $A$ é crescente, $\op\notin A$ e $\opd\in A$, então há elos $e$ tais que
$\op(e)=0$ mas $\opd(e)=1$, isto é, $p\leq Z_e<p+\d$. Denote por $\epd$ o
conjunto de tais elos. A probabilidade de que $|\epd|\geq2$ é $o(\d)$.
Por outro lado, se $\op\notin A$, $\opd\in A$ e $|\epd|=1$, então o (estado do) elo $e$ em questão deve ser {\em
essencial} em $\op$ 
para (a ocorrência ou não de) $A$ no sentido de que $\op\notin A$, mas $\op'\in A$, onde $\op'$ é a
configuração obtida de $\op$ trocando o status do elo $e$ de $0$ para $1$. 
A última probabilidade em (\ref{eq:r1}) fica, então,
$$\pp(\op\notin A, \opd\in A,|\epd|=1)+o(\d).$$

A probabilidade acima pode ser escrita como
$$
\sum_{e\in\g}\pp(\op\notin A, \opd\in A,\epd=\{e\}).
$$

O evento dentro da probabilidade na soma é equivalente ao evento
$$
\{\mbox{$e$ é essencial em $\op$ para $A$},p\leq Z_e<p+\d,\epd=\{e\}\}.
$$
Aquela probabilidade pode ser escrita, então, como
\beqn
\label{eq:piv1}
\mbox{}\!\!\!\!\!\!\!\!&&\pp[\mbox{$e$ é essencial em $\op$ para $A$},p\leq
Z_e<p+\d]\\
\label{eq:piv2}
\mbox{}\!\!\!\!\!\!\!\!&-&\pp[\mbox{$e$ é essencial em $\op$ para $A$},p\leq
Z_e<p+\d,\epd\ne\{e\}].
\eeqn

A probabilidade em~(\ref{eq:piv2}) é limitada superiormente por
$$\pp(|\epd|\geq2)=o(\d).$$

Observemos agora que o evento \{$e$ é essencial em $\op$ para $A$\} não depende de $e$.
Logo, a probabilidade em~(\ref{eq:piv1}) fatora. 

Combinando os argumentos acima, temos
\beqnn
\ppd(A)-\p(A)\=\sum_e\pp(\mbox{$e$ é essencial em $\op$ para $A$})\pp(p\leq Z_e<p+\d)+o(\d)\\
             \=\d\sum_e\pp(\mbox{$e$ é essencial em $\op$ para $A$})+o(\d)\\
             \=\d E(N(A))+o(\d),
\eeqnn
onde $N(A)$ denota o número de elos essenciais em $\op$ para $A$ (e
$E$ é a esperança com respeito a $\pp$).

Modificando um pouco a terminologia, e voltando ao modelo com as variáveis de
Bernoulli, dado um evento qualquer $A\in\e$ e uma configuração $\om\in\Om$, 
definimos um elo $e$ como {\em pivotal} para $A$ (mais precisamente, para
($A,\om$)) se, denotando por
$\om'$ a configuração idêntica a $\om$ em todos os elos com exceção de $e$, em
que $\om$ e $\om'$ são diferentes, uma das duas coisas acontece: ou
$$\om\in A\quad\mbox{e}\quad\om'\notin A$$
ou
$$\om\notin A\quad\mbox{e}\quad\om'\in A.$$

Seja $N(A)$ o número de elos pivotais para $A$. O argumento acima prova o seguinte.

\vs

\bteo {\bf Fórmula de Russo~\cite{kn:R}}

Se $A$ for um evento crescente dependendo de um conjunto finito de elos,
então
\beq
\label{eq:rus}
\frac{d}{dp}\p(A)=\ep(N(A)).
\eeq
\eteo  

\vs

A equação (\ref{eq:rus}) também pode ser escrita
$$
\frac{d}{dp}\p(A)=\sum_{e}\p(\mbox{$e$ é pivotal para $A$}).
$$

O uso que se fará da fórmula de Russo parte da observação de
que o evento \{$e$ é pivotal para $A$\} é independente do elo $e$ e portanto
independente do evento \{$e$ está aberto\}, para deduzir que
$$
\p(\mbox{$e$ é pivotal para $A$})=
\frac{1}{p}\p(\mbox{$e$ está aberto e é pivotal para $A$}).
$$

Logo, se $A$ for crescente, aplicando a fórmula de Russo temos
\beqn
\frac{d}{dp}\p(A)\=\frac{1}{p}\sum_{e}\p(\mbox{$e$ está aberto e é pivotal para $A$})\\
\=\frac{1}{p}\sum_{e}\p(A\cap\{\mbox{$e$ é pivotal para $A$}\})\\
\=\frac{1}{p}\sum_{e}\p(A)\p(\mbox{$e$ é pivotal para $A$}|A)\\
\=\frac{1}{p}\ep(N(A)|A)\p(A).
\eeqn

Dividindo a primeira e última expressões por $\p(A)$ e integrando em
$[p_1,p_2]$ ($0< p_1\leq p_2\leq1$), chegamos a
\beq
P_{p_2}(A)=P_{p_1}(A)\exp\left(\int_{p_1}^{p_2}\frac{1}{p}\ep(N(A)|A)dp\right).
\eeq

A identidade acima será aplicada no próximo capítulo.